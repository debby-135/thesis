\documentclass{wissdoc}
% Autor: Roland Bless 1996-2009, bless <at> kit.edu
% ----------------------------------------------------------------
% Diplomarbeit - Hauptdokument
% ----------------------------------------------------------------
%%
%% $Id: thesis.tex 65 2012-05-10 10:32:11Z bless $
%%
% wissdoc Optionen: draft, relaxed, pdf --> siehe wissdoc.cls
% ------------------------------------------------------------------
% Weitere packages: (Dokumentation dazu durch "latex <package>.dtx")
\usepackage[numbers,sort&compress]{natbib}
% \usepackage{varioref}
% \usepackage{verbatim}
% \usepackage{float}    %z.B. \floatstyle{ruled}\restylefloat{figure}
% \usepackage{subfigure}
% \usepackage{fancybox} % f�r schattierte,ovale Boxen etc.
% \usepackage{tabularx} % automatische Spaltenbreite
% \usepackage{supertab} % mehrseitige Tabellen
% \usepackage[svnon,svnfoot]{svnver} % SVN Versionsinformation 
%% ---------------- end of usepackages -------------

%\svnversion{$Id: thesis.tex 65 2012-05-10 10:32:11Z bless $} % In case that you want to include version information in the footer

%% Informationen f�r die PDF-Datei
\hypersetup{
 pdfauthor={N.N.},
 pdftitle={Not set}
 pdfsubject={Not set},
 pdfkeywords={Not set}
}

% Macros, nicht unbedingt notwendig
%%%%%%%%%%%%%%%%%%%%%%%%%%%%%%%%%%%%%%%%%%%%%%%%%%%%%%%%%%
% macros.tex -- einige mehr oder weniger nuetzliche Makros
% Autor: Roland Bless 1998
%%%%%%%%%%%%%%%%%%%%%%%%%%%%%%%%%%%%%%%%%%%%%%%%%%%%%%%%%%
% $Id: macros.tex 33 2007-01-23 09:00:59Z bless $
%%%%%%%%%%%%%%%%%%%%%%%%%%%%%%%%%%%%%%%%%%%%%%%%%%%%%%%%%%


%%%%%%%%%%%%%%%%%%%%%%%
% Kommentare 
%%%%%%%%%%%%%%%%%%%%%%%
\ifnotdraftelse{
\newcommand{\Kommentar}[1]{}
}{\newcommand{\Kommentar}[1]{{\em #1}}}
% Alles innerhalb von \Hide{} oder \ignore{} 
% wird von LaTeX komplett ignoriert (wie ein Kommentar)
\newcommand{\Hide}[1]{}
\let\ignore\Hide

%%%%%%%%%%%%%%%%%%%%%%%%%
% Leere Seite ohne Seitennummer, wird aber gezaehlt
%%%%%%%%%%%%%%%%%%%%%%%%%

\newcommand{\leereseite}{% Leerseite ohne Seitennummer, n�chste Seite rechts (wenn 2-seitig)
 \clearpage{\pagestyle{empty}\cleardoublepage}
}
%%%%%%%%%%%%%%%%%%%%%%%%%%
% Flattersatz rechts und Silbentrennung, Leerraum nach rechts maximal 1cm
%%%%%%%%%%%%%%%%%%%%%%%%%%
\makeatletter
\newcommand{\myraggedright}{%
 \let\\\@centercr\@rightskip 0pt plus 1cm
 \rightskip\@rightskip
  \leftskip\z@skip
  \parindent\z@
  \spaceskip=.3333em
  \xspaceskip=.5em}
\makeatother

\makeatletter
\newcommand{\mynewline}{%
 \@centercr\@rightskip 0pt plus 1cm
}
\makeatother


%%%%%%%%%%%%%%%%%%%%%%%%%%
% F�r Index
%%%%%%%%%%%%%%%%%%%%%%%%%%
\makeatletter
\def\mydotfill{\leavevmode\xleaders\hb@xt@ .44em{\hss.\hss}\hfill\kern\z@}
\makeatother
\def\bold#1{{\bfseries #1}}
\newbox\dbox \setbox\dbox=\hbox to .4em{\hss.\hss} % dot box for leaders
\newskip\rrskipb \rrskipb=.5em plus3em % ragged right space before break
\newskip\rrskipa \rrskipa=-.17em plus -3em minus.11em % ditto, after
\newskip\rlskipa \rlskipa=0pt plus3em % ragged left space after break
\newskip\rlskipb \rlskipb=.33em plus-3em minus.11em % ragged left before break
\newskip\lskip \lskip=3.3\wd\dbox plus1fil minus.3\wd\dbox % for leaders
\newskip \lskipa \lskipa=-2.67em plus -3em minus.11em %after leaders
\mathchardef\rlpen=1000 \mathchardef\leadpen=600
\def\rrspace{\nobreak\hskip\rrskipb\penalty0\hskip\rrskipa}
\def\rlspace{\penalty\rlpen\hskip\rlskipb\vadjust{}\nobreak\hskip\rlskipa}
\let\indexbreak\rlspace
\def\raggedurl{\penalty10000 \hskip.5em plus15em \penalty0 \hskip-.17em plus-15em minus.11em}
\def\raggeditems{\nobreak\hskip\rrskipb \penalty\leadpen \hskip\rrskipa %
\vadjust{}\nobreak\leaders\copy\dbox\hskip\lskip %
\kern3em \penalty\leadpen \hskip\lskipa %
\vadjust{}\nobreak\hskip\rlskipa}
\renewcommand*\see[2]{\rlspace\emph{\seename}~#1} % from makeidx.sty

%%%%%%%%%%%%%%%%%%%%%%%%%%
% Neue Seite rechts, leere linke Seite ohne Headings
%%%%%%%%%%%%%%%%%%%%%%%%%%
\newcommand{\xcleardoublepage}
{{\pagestyle{empty}\cleardoublepage}}

%%%%%%%%%%%%%%%%%%%%%%%%%%
% Tabellenspaltentypen (benoetigt colortbl)
%%%%%%%%%%%%%%%%%%%%%%%%%%
\newcommand{\PBS}[1]{\let\temp=\\#1\let\\=\temp}
\newcolumntype{y}{>{\PBS{\raggedright\hspace{0pt}}}p{1.35cm}}
\newcolumntype{z}{>{\PBS{\raggedright\hspace{0pt}}}p{2.5cm}}
\newcolumntype{q}{>{\PBS{\raggedright\hspace{0pt}}}p{6.5cm}}
\newcolumntype{g}{>{\columncolor[gray]{0.8}}c} % Grau
\newcolumntype{G}{>{\columncolor[gray]{0.9}}c} % helleres Grau

%%%%%%%%%%%%%%%%%%%%%%%%%%
% Anf�hrungszeichen oben und unten
%%%%%%%%%%%%%%%%%%%%%%%%%%
\newcommand{\anf}[1]{"`{#1}"'}

%%%%%%%%%%%%%%%%%%%%%%%%%%
% Tiefstellen von Text
%%%%%%%%%%%%%%%%%%%%%%%%%%
% S\tl{0} setzt die 0 unter das S (ohne Mathemodus!)
% zum Hochstellen gibt es uebrigens \textsuperscript
\makeatletter
\DeclareRobustCommand*\textlowerscript[1]{%
  \@textlowerscript{\selectfont#1}}
\def\@textlowerscript#1{%
  {\m@th\ensuremath{_{\mbox{\fontsize\sf@size\z@#1}}}}}
\let\tl\textlowerscript
\let\ts\textsuperscript
\makeatother

%%%%%%%%%%%%%%%%%%%%%%%%%%
% Gau�-Klammern
%%%%%%%%%%%%%%%%%%%%%%%%%%
\newcommand{\ceil}[1]{\lceil{#1}\rceil}
\newcommand{\floor}[1]{\lfloor{#1}\rfloor}

%%%%%%%%%%%%%%%%%%%%%%%%%%
% Average Operator (analog zu min, max)
%%%%%%%%%%%%%%%%%%%%%%%%%%
\def\avg{\mathop{\mathgroup\symoperators avg}}

%%%%%%%%%%%%%%%%%%%%%%%%%%
% Wortabk�rzungen
%%%%%%%%%%%%%%%%%%%%%%%%%%
\def\zB{z.\,B.\ }
\def\dh{d.\,h.\ }
\def\ua{u.\,a.\ }
\def\su{s.\,u.\ }
\newcommand{\bzw}{bzw.\ }

%%%%%%%%%%%%%%%%%%%%%%%%%%%%%%%%%%%
% Einbinden von Graphiken
%%%%%%%%%%%%%%%%%%%%%%%%%%%%%%%%%%%
% global scaling factor
\def\gsf{0.9}
%% Graphik, 
%% 3 Argumente: Datei, Label, Unterschrift
\newcommand{\Abbildung}[3]{%
\begin{figure}[tbh] %
\centerline{\scalebox{\gsf}{\includegraphics*{#1}}} %
\caption{#3} %
\label{#2} %
\end{figure} %
}
\let\Abb\Abbildung
%% Abbps
%% Graphik, skaliert, Angabe der Position
%% 5 Argumente: Position, Breite (0 bis 1.0), Datei, Label, Unterschrift
\newcommand{\Abbildungps}[5]{%
\begin{figure}[#1]%
\begin{center}
\scalebox{\gsf}{\includegraphics*[width=#2\textwidth]{#3}}%
\caption{#5}%
\label{#4}%
\end{center}
\end{figure}%
}
\let\Abbps\Abbildungps
%% Graphik, Angabe der Position, frei w�hlbares Argument f�r includegraphics
%% 5 Argumente: Position, Optionen, Datei, Label, Unterschrift
\newcommand{\Abbildungpf}[5]{%
\begin{figure}[#1]%
\begin{center}
\scalebox{\gsf}{\includegraphics*[#2]{#3}}%
\caption{#5}%
\label{#4}%
\end{center}
\end{figure}%
}
\let\Abbpf\Abbildungpf

%%
% Anmerkung: \resizebox{x}{y}{box} skaliert die box auf Breite x und H�he y,
%            ist x oder y ein !, dann wird das uspr�ngliche 
%            Seitenverh�ltnis beibehalten.
%            \rescalebox funktioniert �hnlich, nur das dort ein Faktor
%            statt einer Dimension angegeben wird.
%%
% \Abbps{Position}{Breite in Bruchteilen der Textbreite}{Dateiname}{Label}{Bildunterschrift}
%

\newcommand{\refAbb}[1]{%
s.~Abbildung \ref{#1}}

%%%%%%%%%%%%%%%%%%%%
%% end of macros.tex
%%%%%%%%%%%%%%%%%%%%

% Print URLs not in Typewriter Font
\def\UrlFont{\rm}

\newcommand{\blankpage}{% Leerseite ohne Seitennummer, n�chste Seite rechts
 \clearpage{\pagestyle{empty}\cleardoublepage}
}

%% Einstellungen f�r das gesamte Dokument

% Trennhilfen
% Wichtig! 
% Im ngerman-paket sind zus�tzlich folgende Trennhinweise enthalten:
% "- = zus�tzliche Trennstelle
% "| = Vermeidung von Ligaturen und m�gliche Trennung (bsp: Schaf"|fell)
% "~ = Bindestrich an dem keine Trennung erlaubt ist (bsp: bergauf und "~ab)
% "= = Bindestrich bei dem Worte vor und dahinter getrennt werden d�rfen
% "" = Trennstelle ohne Erzeugung eines Trennstrichs (bsp: und/""oder)

% Trennhinweise fuer Woerter hier beschreiben
\hyphenation{
% Pro-to-koll-in-stan-zen
% Ma-na-ge-ment  Netz-werk-ele-men-ten
% Netz-werk Netz-werk-re-ser-vie-rung
% Netz-werk-adap-ter Fein-ju-stier-ung
% Da-ten-strom-spe-zi-fi-ka-tion Pa-ket-rumpf
% Kon-troll-in-stanz
}

% Index-Datei �ffnen
\ifnotdraft{\makeindex}
%%%%%%%%%%%%%% includeonly %%%%%%%%%%%%%%%%%%%
% Es werden nur die Teile eingebunden, die hier 
% aufgefuehrt sind!
\includeonly{%
titelseite,%
erklaerung,% Ist in KA Pflicht f�r Diplomarbeiten
einleitung,% Motivation, Zielsetzung, Gliederung
grundlagen,% Grundlagen 
analyse,   % Problembeschreibung (Detail) und Related Work
entwurf,   % Beschreibung der Probleml�sung (Konzepte, allg. Architektur, ...)
implemen,  % Beschreibung der Umsetzung/Implementierung
eval,      % Nachweis und Auswertung
zusammenf  % Zusammenfassung der Ergebnisse und Ausblick
}
%%%%%%%%%%%%%%%%%%%%%%%%%%%%%%%%%%%%%%%%%%%%%%
\begin{document}

\frontmatter
\pagenumbering{roman}
\ifnotdraft{
 %% Titelseite
%% Vorlage $Id: titelseite.tex 61 2012-05-03 13:58:03Z bless $

\def\usesf{}
\let\usesf\sffamily % diese Zeile auskommentieren f�r normalen TeX Font

\newsavebox{\Erstgutachter}
\savebox{\Erstgutachter}{\usesf Prof.~Dr.~?.~?????????}
\newsavebox{\Zweitgutachter}
\savebox{\Zweitgutachter}{\usesf Prof.~Dr.~?.~?????????}

\begin{titlepage}
\setlength{\unitlength}{1pt}
\begin{picture}(0,0)(85,770)
\includegraphics[width=\paperwidth]{logos/KIT_Deckblatt}
\end{picture}

\thispagestyle{empty}

%\begin{titlepage}
%%\let\footnotesize\small \let\footnoterule\relax
\begin{center}
\hbox{}
\vfill
{\usesf
{\huge\bfseries Hier steht der Titel der\\
                Diplom-/Studien-/Master-/""Bachelorarbeit \par}
\vskip 1.8cm
Diplomarbeit/Studienarbeit/Masterarbeit/Bachelorarbeit\\
von\\[2mm]
\vskip 1cm

{\large\bfseries Vorname Nachname\\}
\vskip 1.2cm
am Institut f�r Telematik\\
der Fakult�t f�r Informatik\\
%Universit�t Karlsruhe (TH)\\[2ex]
\vskip 3cm
\begin{tabular}{p{5.5cm}l}
Erstgutachter: & \usebox{\Erstgutachter} \\
Zweitgutachter: & \usebox{\Zweitgutachter} \\
Betreuender~Mitarbeiter: & Dipl.-Inform.~?.~????????? \\
\end{tabular}
\vskip 3cm
Bearbeitungszeit:\qquad ??.~Monat~20?? -- ??.~Monat~20??
}
\end{center}
\vfill
\end{titlepage}
%% Titelseite Ende


%%% Local Variables: 
%%% mode: latex
%%% TeX-master: "thesis"
%%% End: 

 \blankpage % Leerseite auf Titelr�ckseite
 %
 % Die folgende Erkl�rung ist f�r Diplomarbeiten Pflicht
 % (siehe Pr�fungsordnung), f�r Studienarbeiten nicht notwendig
 \thispagestyle{empty}
\vspace*{42\baselineskip}
\hbox to \textwidth{\hrulefill}
\par
Ich erkl�re hiermit, dass ich die vorliegende Arbeit selbst�ndig verfasst und
keine anderen als die angegebenen Quellen und Hilfsmittel verwendet habe.

T�bingen, den ??. ?????? 2013

%%%%%%%%%%%%%%%%%%%%%%%%%%%%%%%%%%%%%%%%%%%%%%%%%%%%%%%%%%%%%%%%%%%%%%%%
%% Hinweis:
%%
%% Diese Erkl�rung wird von der Pr�fungsordnung f�r Diplomarbeiten 
%% verlangt und ist zu unterschreiben. F�r Studienarbeiten ist diese
%% Erkl�rung nicht zwingend notwendig, schadet aber auch nicht.
%%%%%%%%%%%%%%%%%%%%%%%%%%%%%%%%%%%%%%%%%%%%%%%%%%%%%%%%%%%%%%%%%%%%%%%%
\clearpage







 \blankpage % Leerseite auf Erkl�rungsr�ckseite
}
%
%% *************** Hier geht's ab ****************
%% ++++++++++++++++++++++++++++++++++++++++++
%% Verzeichnisse
%% ++++++++++++++++++++++++++++++++++++++++++
\ifnotdraft{
{\parskip 0pt\tableofcontents} % toc bitte einzeilig
\blankpage
%\listoffigures
%\blankpage
%\listoftables
%\blankpage
}


%% ++++++++++++++++++++++++++++++++++++++++++
%% Hauptteil
%% ++++++++++++++++++++++++++++++++++++++++++
\graphicspath{{Bilder/}}

\mainmatter
\pagenumbering{arabic}
%% Einleitung.tex
%% $Id: einleitung.tex 61 2012-05-03 13:58:03Z bless $
%%

\chapter{Einleitung}
\label{ch:Einleitung}
%% ==============================
Hinweis: In die Einleitung geh�rt die Motivation ��� 
und Einleitung in die Problemstellung. Die Problemstellung
kann in der Analyse noch detaillierter beschrieben werden.

Bla fasel\ldots



%% ==============================
\section{Zielsetzung der Arbeit}
%% ==============================
\label{ch:Einleitung:sec:Zielsetzung}

Was ist die Aufgabe der Arbeit?

Bla fasel\ldots

%% ==============================
\section{Gliederung der Arbeit}
%% ==============================
\label{ch:Einleitung:sec:Gliederung}

Was enthalten die weiteren Kapitel?

Bla fasel\ldots

%%% Local Variables: 
%%% mode: latex
%%% TeX-master: "thesis"
%%% End: 
  % Einleitung
%% grundlagen.tex
%% $Id: grundlagen.tex 61 2012-05-03 13:58:03Z bless $
%%

\chapter{Grundlagen}
\label{ch:Grundlagen}
%% ==============================
Die Grundlagen m�ssen soweit beschrieben
werden, dass ein Leser das Problem und
die Probleml�sung  versteht.Um nicht zuviel 
zu beschreiben, kann man das auch erst gegen 
Ende der Arbeit schreiben.

Bla fasel\ldots

%% ==============================
\section{Abschnitt 1}
%% ==============================
\label{ch:Grundlagen:sec:Abschnitt1}

Bla fasel\ldots

%% ==============================
\section{Abschnitt 2}
%% ==============================
\label{ch:Grundlagen:sec:Abschnitt2}

Bla fasel\ldots

%% ==============================
\section{Verwandte Arbeiten}
%% ==============================
\label{ch:Grundlagen:sec:RelatedWork}
Hier kommt "`Related Work"' rein.
Eine Literaturrecherche sollte so vollst�ndig wie m�glich sein,
relevante Ans�tze m�ssen beschrieben werden und es sollte deutlich 
gemacht werden, wo diese Ans�tze Defizite aufweisen oder nicht
anwendbar sind, z.\,B. weil sie von anderen Umgebungen oder 
Voraussetzungen ausgehen.


Bla fasel\ldots

%%% Local Variables: 
%%% mode: latex
%%% TeX-master: "thesis"
%%% End: 
  % Grundlagen
%% analyse.tex
%% $Id: analyse.tex 61 2012-05-03 13:58:03Z bless $

\chapter{Analyse}
\label{ch:Analyse}
%% ==============================
In diesem Kapitel sollten zun�chst das zu l�sende Problem
sowie die Anforderungen und die Randbedingungen 
einer L�sung\index{L�sung} beschrieben werden (also nochmal
eine pr�zisierte Aufgabenstellung\index{Aufgabenstellung}).

Dann folgt �blicherweise ein �berblick �ber bereits existierende
L�sungen bzw. Ans�tze, die meistens andere Voraussetzungen bzw.
Randbedingungen annehmen.

Bla fasel\ldots

%% ==============================
\section{Anforderungen}
%% ==============================
\label{ch:Analyse:sec:Anforderungen}
Anforderungen und Randbedingungen\index{Randbedingungen} \ldots

%% ==============================
\section{Existierende L�sungsans�tze}
%% ==============================
\label{ch:Analyse:sec:RelatedWork}

Hier kommt eine ausf�hrliche Diskussion
von "`Related Work"'.

Bla fasel\ldots

%% ==============================
\section{Weiterer Abschnitt}
%% ==============================
\label{ch:Analyse:sec:Abschnitt}

Bla fasel\ldots hat auch schon \cite{TB2000} gesagt und
\cite{TB98,JSAC96,qosr} sollte man mal gelesen haben.
Abbildung~\ref{fig:test} auf S.~\pageref{fig:test} sollte man
sich mal anschauen.

Blindtext Blindtext Blindtext Blindtext Blindtext Blindtext Blindtext
Blindtext Blindtext Blindtext Blindtext Blindtext Blindtext Blindtext
Blindtext Blindtext Blindtext Blindtext Blindtext Blindtext Blindtext
Blindtext Blindtext Blindtext Blindtext Blindtext Blindtext Blindtext
Blindtext Blindtext Blindtext Blindtext Blindtext Blindtext Blindtext
Blindtext Blindtext Blindtext Blindtext Blindtext Blindtext Blindtext
Blindtext Blindtext Blindtext Blindtext Blindtext Blindtext Blindtext
Blindtext Blindtext Blindtext Blindtext Blindtext Blindtext Blindtext
Blindtext Blindtext Blindtext Blindtext Blindtext Blindtext Blindtext

Blindtext Blindtext Blindtext Blindtext Blindtext Blindtext Blindtext
Blindtext Blindtext Blindtext Blindtext Blindtext Blindtext Blindtext
Blindtext Blindtext Blindtext Blindtext Blindtext Blindtext Blindtext
Blindtext Blindtext Blindtext Blindtext Blindtext Blindtext Blindtext
Blindtext Blindtext Blindtext Blindtext Blindtext Blindtext Blindtext
Blindtext Blindtext Blindtext Blindtext Blindtext Blindtext Blindtext
Blindtext Blindtext Blindtext Blindtext Blindtext Blindtext Blindtext
Blindtext Blindtext Blindtext Blindtext Blindtext Blindtext Blindtext
Blindtext Blindtext Blindtext Blindtext Blindtext Blindtext Blindtext
Blindtext Blindtext Blindtext Blindtext Blindtext Blindtext Blindtext
Blindtext Blindtext Blindtext Blindtext Blindtext Blindtext Blindtext
Blindtext Blindtext Blindtext Blindtext Blindtext Blindtext Blindtext
Blindtext Blindtext Blindtext Blindtext Blindtext Blindtext Blindtext

Blindtext Blindtext Blindtext Blindtext Blindtext Blindtext Blindtext
Blindtext Blindtext Blindtext Blindtext Blindtext Blindtext Blindtext
Blindtext Blindtext Blindtext Blindtext Blindtext Blindtext Blindtext
Blindtext Blindtext Blindtext Blindtext Blindtext Blindtext Blindtext
Blindtext Blindtext Blindtext Blindtext Blindtext Blindtext Blindtext
Blindtext Blindtext Blindtext Blindtext Blindtext Blindtext Blindtext
Blindtext Blindtext Blindtext Blindtext Blindtext Blindtext Blindtext
Blindtext Blindtext Blindtext Blindtext Blindtext Blindtext Blindtext
Blindtext Blindtext Blindtext Blindtext Blindtext Blindtext Blindtext
Blindtext Blindtext Blindtext Blindtext Blindtext Blindtext Blindtext

Blindtext Blindtext Blindtext Blindtext Blindtext Blindtext Blindtext
Blindtext Blindtext Blindtext Blindtext Blindtext Blindtext Blindtext
Blindtext Blindtext Blindtext Blindtext Blindtext Blindtext Blindtext
Blindtext Blindtext Blindtext Blindtext Blindtext Blindtext Blindtext
Blindtext Blindtext Blindtext Blindtext Blindtext Blindtext Blindtext
Blindtext Blindtext Blindtext Blindtext Blindtext Blindtext Blindtext
Blindtext Blindtext Blindtext Blindtext Blindtext Blindtext Blindtext
Blindtext Blindtext Blindtext\index{Blindtext} Blindtext Blindtext Blindtext Blindtext

\begin{figure}[!htbp]
  \centering
  \fbox{\parbox{0.8\textwidth}{
  Abbildungen sollten m�glichst als EPS (Encapsulated Postscript) 
  bzw. PDF eingebunden werden.
  Zur Erzeugung sauberer EPS-Dateien empfiehlt sich das Tool \texttt{ps2eps}
  zur Nachbearbeitung von Postscript-Dateien. Mit \texttt{epstopdf} kann
  dann eine PDF-Datei zum Einbinden erzeugt werden.}}
  \caption{Testabbildung}
  \label{fig:test}
\end{figure}

Blindtext Blindtext Blindtext Blindtext Blindtext Blindtext Blindtext
Blindtext Blindtext Blindtext Blindtext Blindtext Blindtext Blindtext
Blindtext Blindtext Blindtext Blindtext Blindtext Blindtext Blindtext
Blindtext Blindtext Blindtext Blindtext Blindtext Blindtext Blindtext
Blindtext Blindtext Blindtext Blindtext Blindtext Blindtext Blindtext
Blindtext Blindtext Blindtext Blindtext Blindtext Blindtext Blindtext
Blindtext Blindtext Blindtext Blindtext Blindtext Blindtext Blindtext

Blindtext Blindtext Blindtext Blindtext Blindtext Blindtext Blindtext
Blindtext Blindtext Blindtext Blindtext Blindtext Blindtext Blindtext
Blindtext Blindtext Blindtext Blindtext Blindtext Blindtext Blindtext

Blindtext Blindtext Blindtext Blindtext Blindtext Blindtext Blindtext
Blindtext Blindtext Blindtext Blindtext Blindtext Blindtext Blindtext
Blindtext Blindtext Blindtext Blindtext Blindtext Blindtext Blindtext
Blindtext Blindtext Blindtext Blindtext Blindtext Blindtext Blindtext
Blindtext Blindtext Blindtext Blindtext Blindtext Blindtext Blindtext
Blindtext Blindtext Blindtext Blindtext Blindtext Blindtext Blindtext

Blindtext Blindtext Blindtext Blindtext Blindtext Blindtext Blindtext
Blindtext Blindtext Blindtext Blindtext Blindtext Blindtext Blindtext
Blindtext Blindtext Blindtext Blindtext Blindtext Blindtext Blindtext
Blindtext Blindtext Blindtext Blindtext Blindtext Blindtext Blindtext
Blindtext Blindtext Blindtext Blindtext Blindtext Blindtext Blindtext
Blindtext Blindtext Blindtext Blindtext Blindtext Blindtext Blindtext
Blindtext Blindtext Blindtext Blindtext Blindtext Blindtext Blindtext
Blindtext Blindtext Blindtext Blindtext Blindtext Blindtext Blindtext
Blindtext Blindtext Blindtext Blindtext Blindtext Blindtext Blindtext
Blindtext Blindtext Blindtext Blindtext Blindtext Blindtext Blindtext
Blindtext Blindtext Blindtext Blindtext Blindtext Blindtext Blindtext
Blindtext Blindtext Blindtext Blindtext Blindtext Blindtext Blindtext
Blindtext Blindtext Blindtext Blindtext Blindtext Blindtext Blindtext
Blindtext Blindtext Blindtext Blindtext Blindtext Blindtext Blindtext

Blindtext Blindtext Blindtext Blindtext Blindtext Blindtext Blindtext
Blindtext Blindtext Blindtext Blindtext Blindtext Blindtext Blindtext
Blindtext Blindtext Blindtext Blindtext Blindtext Blindtext Blindtext
Blindtext Blindtext Blindtext Blindtext Blindtext Blindtext Blindtext
Blindtext Blindtext Blindtext Blindtext Blindtext Blindtext Blindtext
Blindtext Blindtext Blindtext Blindtext Blindtext Blindtext Blindtext
Blindtext Blindtext Blindtext Blindtext Blindtext Blindtext Blindtext
Blindtext Blindtext Blindtext Blindtext Blindtext Blindtext Blindtext
Blindtext Blindtext Blindtext Blindtext Blindtext Blindtext Blindtext
Blindtext Blindtext Blindtext Blindtext Blindtext Blindtext Blindtext
Blindtext Blindtext Blindtext Blindtext Blindtext Blindtext Blindtext
%% ==============================
\section{Zusammenfassung}
%% ==============================
\label{ch:Analyse:sec:zusammenfassung}

Am Ende sollten ggf. die wichtigsten Ergebnisse nochmal in \emph{einem}
kurzen Absatz zusammengefasst werden.

%%% Local Variables: 
%%% mode: latex
%%% TeX-master: "thesis"
%%% End: 
     % Analyse
%% entwurf.tex
%% $Id: entwurf.tex 61 2012-05-03 13:58:03Z bless $
%%

\chapter{Entwurf}
\label{ch:Entwurf}
%% ==============================
In diesem Kapitel erfolgt die ausf�hrliche Beschreibung des eigenen
L�sungsansatzes. Dabei sollten L�sungsalternativen diskutiert und
Entwurfsentscheidungen dargelegt werden.


Bla fasel\ldots

%% ==============================
\section{Abschnitt 1}
%% ==============================
\label{ch:Entwurf:sec:Abschnitt1}

Bla fasel\ldots

%% ==============================
\section{Abschnitt 2}
%% ==============================
\label{ch:Entwurf:sec:Abschnitt2}

Bla fasel\ldots

text Blindtext
Blindtext Blindtext Blindtext Blindtext Blindtext Blindtext Blindtext
Blindtext Blindtext Blindtext Blindtext Blindtext Blindtext Blindtext
Blindtext Blindtext Blindtext Blindtext Blindtext Blindtext Blindtext
Blindtext Blindtext Blindtext Blindtext Blindtext Blindtext Blindtext
Blindtext Blindtext Blindtext Blindtext Blindtext Blindtext Blindtext
Blindtext Blindtext Blindtext Blindtext Blindtext Blindtext Blindtext
Blindtext Blindtext Blindtext Blindtext Blindtext Blindtext Blindtext
Blindtext Blindtext Blindtext Blindtext Blindtext Blindtext Blindtext
Blindtext Blindtext Blindtext Blindtext Blindtext Blindtext Blindtext
Blindtext Blindtext Blindtext Blindtext Blindtext Blindtext Blindtext
Blindtext Blindtext Blindtext Blindtext Blindtext Blindtext Blindtext
Blindtext Blindtext Blindtext Blindtext Blindtext Blindtext Blindtext

Blindtext Blindtext Blindtext Blindtext Blindtext Blindtext Blindtext
Blindtext Blindtext Blindtext Blindtext Blindtext Blindtext Blindtext
Blindtext Blindtext Blindtext Blindtext Blindtext Blindtext Blindtext
Blindtext Blindtext Blindtext Blindtext Blindtext Blindtext Blindtext
Blindtext Blindtext Blindtext Blindtext Blindtext Blindtext Blindtext
Blindtext Blindtext Blindtext Blindtext Blindtext Blindtext Blindtext
Blindtext Blindtext Blindtext Blindtext Blindtext Blindtext Blindtext
Blindtext Blindtext Blindtext Blindtext Blindtext Blindtext Blindtext
Blindtext Blindtext Blindtext Blindtext Blindtext Blindtext Blindtext
Blindtext Blindtext Blindtext Blindtext Blindtext Blindtext Blindtext
Blindtext Blindtext Blindtext Blindtext Blindtext Blindtext Blindtext
Blindtext Blindtext Blindtext Blindtext Blindtext Blindtext Blindtext
Blindtext Blindtext Blindtext Blindtext Blindtext Blindtext Blindtext
Blindtext Blindtext Blindtext Blindtext Blindtext Blindtext Blindtext
Blindtext Blindtext Blindtext Blindtext Blindtext Blindtext Blindtext
Blindtext Blindtext Blindtext Blindtext Blindtext Blindtext Blindtext
Blindtext Blindtext Blindtext Blindtext Blindtext Blindtext Blindtext
Blindtext Blindtext Blindtext Blindtext Blindtext Blindtext Blindtext
Blindtext Blindtext Blindtext Blindtext Blindtext Blindtext Blindtext
Blindtext Blindtext Blindtext Blindtext Blindtext Blindtext Blindtext

Blindtext Blindtext Blindtext Blindtext Blindtext Blindtext Blindtext
Blindtext Blindtext Blindtext Blindtext Blindtext Blindtext Blindtext
Blindtext Blindtext Blindtext Blindtext Blindtext Blindtext Blindtext
Blindtext Blindtext Blindtext Blindtext Blindtext Blindtext Blindtext
Blindtext Blindtext Blindtext Blindtext Blindtext Blindtext Blindtext
Blindtext Blindtext Blindtext Blindtext Blindtext Blindtext Blindtext
Blindtext Blindtext Blindtext Blindtext Blindtext Blindtext Blindtext
Blindtext Blindtext Blindtext Blindtext Blindtext Blindtext Blindtext
Blindtext Blindtext Blindtext Blindtext Blindtext Blindtext Blindtext
Blindtext Blindtext Blindtext Blindtext Blindtext Blindtext Blindtext
Blindtext Blindtext Blindtext Blindtext Blindtext Blindtext Blindtext
Blindtext Blindtext Blindtext Blindtext Blindtext Blindtext Blindtext
Blindtext Blindtext Blindtext Blindtext Blindtext Blindtext Blindtext
Blindtext Blindtext Blindtext Blindtext Blindtext Blindtext Blindtext
Blindtext Blindtext Blindtext Blindtext Blindtext Blindtext Blindtext

Blindtext Blindtext Blindtext Blindtext Blindtext Blindtext Blindtext
Blindtext Blindtext Blindtext Blindtext Blindtext Blindtext Blindtext
Blindtext Blindtext Blindtext Blindtext Blindtext Blindtext Blindtext
Blindtext Blindtext Blindtext Blindtext Blindtext Blindtext Blindtext
Blindtext Blindtext Blindtext Blindtext Blindtext Blindtext Blindtext
Blindtext Blindtext Blindtext Blindtext Blindtext Blindtext Blindtext
Blindtext Blindtext Blindtext Blindtext Blindtext Blindtext Blindtext
Blindtext Blindtext Blindtext Blindtext Blindtext Blindtext Blindtext
Blindtext Blindtext Blindtext Blindtext Blindtext Blindtext Blindtext
Blindtext Blindtext Blindtext Blindtext Blindtext Blindtext Blindtext
Blindtext Blindtext Blindtext Blindtext Blindtext Blindtext Blindtext
Blindtext Blindtext Blindtext Blindtext Blindtext Blindtext Blindtext
Blindtext Blindtext Blindtext Blindtext Blindtext Blindtext Blindtext
Blindtext Blindtext Blindtext Blindtext Blindtext Blindtext Blindtext
Blindtext Blindtext Blindtext Blindtext Blindtext Blindtext Blindtext
Blindtext Blindtext Blindtext Blindtext Blindtext Blindtext Blindtext

Blindtext Blindtext Blindtext Blindtext Blindtext Blindtext Blindtext
Blindtext Blindtext Blindtext Blindtext Blindtext Blindtext Blindtext
Blindtext Blindtext Blindtext Blindtext Blindtext Blindtext Blindtext
Blindtext Blindtext Blindtext Blindtext Blindtext Blindtext Blindtext
Blindtext Blindtext Blindtext Blindtext Blindtext Blindtext Blindtext
Blindtext Blindtext Blindtext Blindtext Blindtext Blindtext Blindtext
Blindtext Blindtext Blindtext Blindtext Blindtext Blindtext Blindtext
Blindtext Blindtext Blindtext Blindtext Blindtext Blindtext Blindtext
Blindtext Blindtext Blindtext Blindtext Blindtext Blindtext Blindtext
Blindtext Blindtext Blindtext Blindtext Blindtext Blindtext Blindtext
Blindtext Blindtext Blindtext Blindtext Blindtext Blindtext Blindtext
Blindtext Blindtext Blindtext Blindtext Blindtext Blindtext Blindtext
Blindtext Blindtext Blindtext Blindtext Blindtext Blindtext Blindtext

%% ==============================
\section{Zusammenfassung}
%% ==============================
\label{ch:Entwurf:sec:zusammenfassung}

Am Ende sollten ggf. die wichtigsten Ergebnisse nochmal in \emph{einem}
kurzen Absatz zusammengefasst werden.

%%% Local Variables: 
%%% mode: latex
%%% TeX-master: "thesis"
%%% End: 
     % Entwurf
%% implemen.tex
%% $Id: implemen.tex 61 2012-05-03 13:58:03Z bless $
%%

\chapter{Implementierung}
\label{ch:Implementierung}
%% ==============================
Bla fasel\ldots

%% ==============================
\section{Abschnitt 1}
%% ==============================
\label{ch:Implementierung:sec:Abschnitt1}

Bla fasel\ldots

%% ==============================
\section{Abschnitt 2}
%% ==============================
\label{ch:Implementierung:sec:Abschnitt2}

Bla fasel\ldots

%%% Local Variables: 
%%% mode: latex
%%% TeX-master: "thesis"
%%% End: 
    % Implementierung
%% eval.tex
%% $Id: eval.tex 61 2012-05-03 13:58:03Z bless $

\chapter{Evaluierung}
\label{ch:Evaluierung}
%% ==============================
Hier kommt der Nachweis, dass das in Kapitel~\ref{ch:Entwurf}
entworfene Konzept auch funktioniert. Leistungsmessungen einer
Implementierung werden auch immer gerne gesehen.

Bla fasel\ldots

%% ==============================
\section{Abschnitt 1}
%% ==============================
\label{ch:Evaluierung:sec:Abschnitt1}

Bla fasel\ldots

%% ==============================
\section{Abschnitt 2}
%% ==============================
\label{ch:Evaluierung:sec:Abschnitt2}

Bla fasel\ldots

%% ==============================
\section{Zusammenfassung}
%% ==============================
\label{ch:Evaluierung:sec:zusammenfassung}

Am Ende sollten ggf. die wichtigsten Ergebnisse nochmal in \emph{einem}
kurzen Absatz zusammengefasst werden.

%%% Local Variables: 
%%% mode: latex
%%% TeX-master: "thesis"
%%% End: 
        % Evaluierung
%% zusammenf.tex
%% $Id: zusammenf.tex 61 2012-05-03 13:58:03Z bless $
%%

\chapter{Zusammenfassung und Ausblick}
\label{ch:Zusammenfassung}
%% ==============================
Bla fasel\ldots

(Keine Untergliederung mehr!)

%%% Local Variables: 
%%% mode: latex
%%% TeX-master: "thesis"
%%% End: 
   % Zusammenfassung und Ausblick

%% ++++++++++++++++++++++++++++++++++++++++++
%% Anhang
%% ++++++++++++++++++++++++++++++++++++++++++

\appendix
%\include{anhang_a}
%\include{anhang_b}

%% ++++++++++++++++++++++++++++++++++++++++++
%% Literatur
%% ++++++++++++++++++++++++++++++++++++++++++
%  mit dem Befehl \nocite werden auch nicht 
%  zitierte Referenzen abgedruckt
\cleardoublepage
\phantomsection
\addcontentsline{toc}{chapter}{\bibname}
%%
\nocite{*} % nur angeben, wenn auch nicht im Text zitierte Quellen 
           % erscheinen sollen
\bibliographystyle{itmabbrv} % mit abgek�rzten Vornamen der Autoren
%\bibliographystyle{gerplain} % abbrvnat unsrtnat
% spezielle Zitierstile: Labels mit vier Buchstaben und Jahreszahl
%\bibliographystyle{itmalpha}  % ausgeschriebene Vornamen der Autoren
\bibliography{thesis}
%% ++++++++++++++++++++++++++++++++++++++++++
%% Index
%% ++++++++++++++++++++++++++++++++++++++++++
\ifnotdraft{
\cleardoublepage
\phantomsection
\printindex            % Index, Stichwortverzeichnis
}
\end{document}
%% end of file
